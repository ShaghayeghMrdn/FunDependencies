\section{Related Work}
\label{sec:related}

Work on improving web performance has been going for more than two decades now. 
Prior work has focused on various components of the overall page load time
from remodifying the source code of the webpage itself, to optimizing the
network component of the over all execution time and more recently some work
has been in improving the computatio time latency. 

Erman et all \cite{} %In Proceedings of the Ninth ACM Conference on Emerging Networking Experiments and Technologies
has shown that unline desktop browsers, optimizations such as SPDY/HTTP@ does not improve
performance of Web pages on mobile browsers. They show that this is because of the negative interactions
between the cellular state machine and the transport protocol. SImilary Qian
et al \cite{ } %Web caching on smartphones: Ideal vs. reality. In MobiSys 
show that caching does not provide
page load improvements for mobile browsers. 
Google already released a paper last year \cite{} %Caching Doesn’t Improve Mobile Web Performance (Much)
which talks of how there is very little improvement to the overall page load time despite 
significant improvements in the caching hit rate. 

Many of the research on explicitly improving mobile browser performance has seen mixed results.
FLywheel \cite{} is Google's compression proxy that compressses web content to significanlty 
reduce the use of expensive cellular data. The authots note that whie Flywheel succeeds in
reducong the data usage, its effect on page load performance is more mixed; it helps
performance of certain pages and hurts performance of others. Flexiweb \cite{} is built over
Google's compression proxy to ensure that the proxy does not hurt page load times. But FlexiWeb
is not designed to explicitly improve page load performance. Wang et al \cite{Wprof} show that
speculative loading in one of only client only approaches that can improve mobile
browser performance. However, speculative loading requires knowledge of what objects are likely
to be requested by the user. 

Other research works have looked at metrics orthognal to the page load time metric. Parcel
\cite{} %Proxy assisted browsing in cellular networks for energy and latency reduction.
 uses a proxy
approach to divide the page load process between the mobile device and the proxy. Becayse Parcel is
a network appraoch, the evaluations are largely with respect to reduction to network
latencies. Klotski \cite{} focusses on increasing the number of objects rendered in the first
5 seconds to improve the user quality of experience. 

Other client side imporvements reduce enrge usage and computational delays using parallel
browsers \cite{} %A Case for Parallelizing Web Pages. In HotPar,  Fast and Parallel Web Page Layout. In WWW.
and improved hardware \cite{} % High-Performance and Energy-Efficient Mobile Web Browsing on Big/Little Systems. In HPCA
 By improving the parallelization 
for necessary page load tasks (eg: rendering) these systems reduce energy usage
and have positive impact on page load times. 

While there has been several recent efforts on improving mobile browser performance
they have not been uniformly successful.

