\section{Motivation}
\label{sec:motivation}


Major web browsers like Chrome, Firefox, and Safari have recently invested a lot of resources,
time and energy into improving web performance on mobile devices, specifically by targeting 
the network usage. However, the network now comprises less than 30\% \cite{njait2016www} of the total critical
path for an average page load on a mobile device. This includes caching almost 95\% of the
resources that are fetched from the server \cite{vesuna2016caching}, DNS presolution, DNS caching, TCP reconnect, etc.
Chrome released a paper last year showing how improved caching algorithms, despite having 
significant improvements on the desktop, don't have proportionate improvements 
on mobile devices. This is primarily attributed to the fact that computation comprises more than 65\%
of the critical path during a page load. This illustrates the need to further optimize the computation 
time. 

During the Chrome dev summit this year, their team announced the latest improvements they have 
made in their browser to improve the page load time. Interestingly, most of their work focuses on
improving the compilation and parsing time by introducing compile and parser cache. 
Recent studies \cite{url4} still report that the median page load time for a mobile website 
is about 14 seconds. Research \cite{url4} shows that a user will only wait for 3 seconds 
before abandoning a web site if it shows no response at all. A lot of prior work \cite {njait2016www}
has been done to compare the page load times on mobile vs desktop, and recent results
from 2016 claim that despite the increasing compute resources in mobile devices,
the computation time on mobile is significantly higher than their desktop counterparts. 
 Our experiments
on the most popular news and sports websites on the latest mobile hardware and the latest 
Chrome version reveal that despite these recent efforts, scripting still takes significantly more
time than any other component. With our understanding of 
computation being the current bottleneck for high page load times, 
we conduct a set of experiments to evaluate the reasons for this
exceedingly high computation time. In order to do this, we break down computation
into four categories: scripting, loading, painting, and rendering. We observe that scripting takes more than 70\% of the total
computation time, which is more than all the other categories combined (Figure 3). 
This makes it all the more important to do an in-depth analysis of the computation time to clearly
understand where exactly this time is being spent. Using results from this study, we would eventually
design our caching framework for the javascript execution output. 
