\section{Motivation}
\label{sec:motivation}


Major web browsers like Chrome, firefox, Safari have recenlty invested a lot of resources,
time and energy in improving the web performance on the mobile, specially by targetting 
the network usage. So much so that the network now comprises less than 30\% \cite{} of the total critical
path for an average page load on the mobile device. This includes caching almost 95\% of the
resource that is fetched from the server \cite{}, dns presolution, dns caching, tcp reconnect etc.
Chrome released a paper last year talking about how improved caching algorithms, despite having 
significant improvements on the desktop, don't have the same proportionate imrpovements 
on the mobile. This is primarily attributed to the fact that computation comprises more than 65\%
of the critical path during a page load. This presses the need to further optimise the compuatation 
time. 

During the Chrome dev summit this year, their team announced the latest improvements they have 
made in their browser to improve the page load time. Interstingly most of their work talks about
improving the compilation and parsing time by introducing compile and parser cache. 
Recent studies \cite{} still report that the median page load time for a mobile website 
is about 14 seconds. Research \cite{} shows that a user will only wait for 3 seconds 
before abandoning a web site if it shows no response at all. A lot of prior work \cite {}
has been done to compare the page load times on mobile vs desktop, and recent results
uptil last year claim that despite the increasing compute resources in mobile devices 
the computation time on mobile is significanlty higher than their desktop counterparts. 
 Our experiments
on the most popular news and sports websites on the latest mobile hardware and the latest 
chrome verison reveal that despite these recent efforts, scripting still takes a significantly more
amount of time as compared to the other components of the total computation time. We break down computation
into four categories: scripting, loading, paitnint, rendering.
and observe that scripting essentially takes more than 70\% of the total
computation time which is more than all the other categories combined. 
This makes it all the more important to do an in depth of analysis of the computation time to clearly
understand where exactly is the time being spent. 
