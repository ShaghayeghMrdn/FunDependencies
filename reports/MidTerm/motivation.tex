\section{Motivation}
\label{sec:motivation}

% Chrome released a paper last year showing how improved caching algorithms, despite having 
% significant improvements on the desktop, don't have the same proportionate improvements 
% on mobile devices. This is primarily attributed to the fact that computation comprises more than 65\%
% of the critical path during a page load. This illustrates the need to further optimize the computation 
% time. 
Computation time is dominant factor in improving PLT.  Recent
work~\cite{chrome} highlights that enhanced caching algorithms
significantly improve performance in desktop environment, where
computation-power is plentiful. In comparison, PLT of mobile devices
is improved by relatively small fraction. The authors describe that
because computation consists at least 65\% of PLT,
resource-constrained mobile devices benefit less from the caching
algorithms. 
% Recent effort from industry has been focused on 
% Major web browsers like Chrome, Firefox, and Safari have recenlty invested a lot of resources,
% time and energy into improving web performance on mobile devices, specifically by targeting 
% the network usage. However, the network now comprises less than 30\% \cite{njait2016www} of the total critical
% path for an average page load on a mobile device. 
% This includes caching almost 95\% of the
% resources that are fetched from the server \cite{vesuna2016caching}, dns presolution, dns caching, tcp reconnect etc.
Also, recent effort from industry to improve mobile web performance
has been focused on reducing network usage, under assumption that
network is dominant factor in PLT. However, recent
study~\cite{njait2016www} shows that the network latency composes of
less than 30\% of the total PLT. The authors note that not only the
performance of mobile network improved but also numerious
optimizations (e.g., resource caching~\cite{vesuna2016caching}, DNS
presolution and caching, TCP Fast reconnection) contribute to reduce
the network latency.
Above prior works suggest that for mobile devices, computation time is
the most dominant factor in total PLT and therefore, reducing
computation time is a key to further reduce PLT.

During the Chrome dev summit this year, their team announced the latest improvements they have 
made in their browser to improve the page load time. Interestingly, most of their work focuses on
improving the compilation and parsing time by introducing compile and parser cache. 
Recent studies \cite{url4} still report that the median page load time for a mobile website 
is about 14 seconds. Research \cite{url4} shows that a user will only wait for 3 seconds 
before abandoning a web site if it shows no response at all. A lot of prior work \cite {njait2016www}
has been done to compare the page load times on mobile vs desktop, and recent results
from 2016 claim that despite the increasing compute resources in mobile devices,
the computation time on mobile is significantly higher than their desktop counterparts. 
 Our experiments
on the most popular news and sports websites on the latest mobile hardware and the latest 
Chrome version reveal that despite these recent efforts, scripting still takes significantly more
time ,as compared to the other components of the total computation time. With our understanding of 
computation being the current bottleneck for high page load times, 
we conduct a set of experiments to evaluate the reasons for this
exceedingly high computation time. In order to do this, we break down computation
into four categories: scripting, loading, painting, and rendering.
and observe that scripting essentially takes more than 70\% of the total
computation time which is more than all the other categories combined (Figure 3). 
This makes it all the more important to do an in-depth analysis of the computation time to clearly
understand where exactly this time is being spent. Using results from this study, we would eventually
design our caching framework for the javascript execution output. 
